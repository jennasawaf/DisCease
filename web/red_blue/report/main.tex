\documentclass[a4paper,12pt]{article}
\usepackage[utf8]{inputenc}
\usepackage{graphicx}
\usepackage{subcaption}
\usepackage{chngcntr}
\counterwithin{figure}{section}
\usepackage{authblk}
\usepackage{titling}
\setlength{\droptitle}{-10em}   % This eliminates the space on top of title
\usepackage[font=scriptsize]{caption}
\usepackage{float}
\usepackage{siunitx}

%opening
\title{Complex Systems And Networks\\Homwork 2}
\author{Akhil Devarashetti\\Jenna Sawaf}
\begin{document}
  \maketitle

  \begin{abstract}
    abstract
  \end{abstract}

  \section{Title}
  \subsection{System Description}
  lorem ispudim
  Your report will consist of two parts:
  1. The first part will compare Policies 1 and 2. You will systematically vary the Policy 2 parameters,
  n and p, over reasonable ranges. I suggest n = [5, 10, 20] and p = [3, 5], which will give you 6
  cases. You can do more, but no less. For each case, you will run at least 30 simulations of at least
  10 epochs each – keep the numbers the same for all cases – for both policies. You will then plot
  the happiness time-series for all cases – the one Policy 1 case, and all the Policy 2 cases – on the
  same plot, so if you tried 6 cases of Policy 2, you will have 7 time-series plots. You may also plot
  anything else you think would be useful (e.g., each case separately against the Policy 1 case,
  either as a time-series or a phase-plot). In the text for this part, you will briefly describe what you
  did, what the results show, what they might mean, and your explanation. Use the plots to make
  your points clearly. The entire report for this part should be no more than 2 pages, and preferably
  just 1 page, including figures and text.
  2. The second part of the report will be like the first part, but will consist of 1-page sections by each
  group member (with names at the top of each page). Each individual’s page should compare their
  Policy 3 with Policy 1. This is also where you will describe your Policy 3 and explain its logic.
  Since the parameters of Policy 3 will be determined by your choice, you should decide how to
  vary them to explore the space of possibilities in a systematic way. Again, you will plot the timeseries for Policy 1 and all the cases of your policy, as well as any other figures that help you
  analyze the outcomes. Your text will follow the same guidelines as part 1, but each individual’s
  part should be exactly 1 page including figures and text. Since the figures have to be clear, you
  will have to come up with the best way to summarize your results.

  \subsection{Title 2}
  lorem ispudim

  \begin{figure}[ht]
    \centering
    \begin{subfigure}[b]{0.49\linewidth}
      \includegraphics[width=\linewidth]{train_set_confusion_matrix.png}
      \caption{Confusion matrix for train set}
      \label{fig:1.1a}
    \end{subfigure}
    \begin{subfigure}[b]{0.49\linewidth}
      \includegraphics[width=\linewidth]{test_set_confusion_matrix.png}
      \caption{Confusion matrix for test set}
      \label{fig:1.1b}
    \end{subfigure}
    \caption{}
  \end{figure}
  \begin{figure}[ht]
    \centering
    \includegraphics[width=0.5\linewidth]{classification_error.png}
    \caption{Error in classifying training samples for every 10th epoch}
    \label{fig:1.2}
  \end{figure}

\end{document}
